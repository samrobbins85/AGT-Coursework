\documentclass[addpoints]{exam}
\usepackage{amsmath}
\usepackage{amssymb}
\usepackage{listings}
\usepackage{pxfonts}
\usepackage{xcolor}
\lstset{language=Python,
	basicstyle=\ttfamily,
	keywordstyle=\bfseries,
	showstringspaces=false,
	morekeywords={if, else, then, print, end, for, do, while,output},
	tabsize=4,
	mathescape=true,
	moredelim=**[is][\color{red}]{@}{@},
}

\pagestyle{headandfoot}
\firstpageheadrule
\runningheadrule
\firstpageheader{Algorithmic Game Theory}{}{mbkb74}
\runningheader{Algorithmic Game Theory}{Individual Component}{mbkb74}
\firstpagefooter{}{}{}
\runningfooter{}{}{}
\renewcommand{\solutiontitle}{\noindent\textbf{Answer:}\par\noindent}


\printanswers
\usepackage{graphicx}
\marksnotpoints
\bracketedpoints
\pointsdroppedatright
\pointsinrightmargin
\begin{document}
\begin{center}
    \LARGE{Algorithmic Game Theory Summative Assignment -- Individual Component}\\[0.1cm]
\end{center}

\begin{questions}

    \question 	      Consider the following instance of the load balancing game where the number of tasks is equal to the number of machines, and in particular we have:
    \begin{itemize}
        \item $m$ identical machines $M_1, M_2, \dots, M_m$ (all of speed 1),
        \item $m$ identical tasks $w_1 = w_2 = \dots = w_m = 1$.
    \end{itemize}
    Consider also the mixed strategy profile $A$ where each of the tasks is assigned to all machines equiprobably (i.e. with probability $1/m$).
    \begin{parts}
        \part[3]Calculate the ratio $cost(A)/cost(OPT)$ in the special case where $m=2$.
        \begin{solution}[2in]
            \begin{itemize}
                \item There are $2^2=4$ possible assignments of 2 tasks to 2 machines
                \item In two of these both tasks are assigned to 1 machine (time 2)
                \item In the other two one task is assigned to each machine (time 1)
                \item The Cost of A is therefore $1/4(2+2+1)=1.5$
                \item However the optimal cost is where one is assigned to each machine $1$
                \item The ratio is therefore 1.5
            \end{itemize}
        \end{solution}
        \part[3]Calculate the ratio $cost(A)/cost(OPT)$ in the special case where $m=3$.
        \begin{solution}[2in]
        \end{solution}
        \part[5]Discuss what this ratio is for arbitrary $m$. What does this imply about the Price of Anarchy on identical machines for mixed Nash equilibria?
        \begin{solution}[2in]
        \end{solution}
    \end{parts}
    \droptotalpoints
    \newpage


\end{questions}





\end{document}




